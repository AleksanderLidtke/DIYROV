\pagestyle{chapter-fancy-style}
\chapter{Conclusions}

%=====================
\section{Summary of the design}

It has been demonstrated that even within a~limited budged of less than £300 and
access only to basic tools, a~fully-functional remotely operated underwater vehicle 
can be designed, manufactured and tested.
The overall design concept using only four degrees of freedom control and employing
a compact vehicle shape has proven to provide sufficient performance that would be
expected of a low-cost sensor platform for use in shallow waters.
Several specific aspects of the design should be refined but a precious
few have also proven to be quite successful. Each of these groups will be discussed
in more detail in the subsequent parts of this chapter.

%=====================
\section{Successful design features}

Several of the chosen engineering solutions have been found to work particularly
well during the tests. These were:
%
\begin{itemize}
\item Using the on-board Arduino micro-controller to allow control of the ROV
	from a standard laptop has proven to be a very flexible and low-cost solution
\item The X-Box controller used to collect input data has been found to provide
	a robust, easy-to-use interface
\item Overall general arrangement of the vehicle has been successful and integrating
	all of the electronics into a removable internal frame allowed ease of access
	during assembly and disassembly
\item Placing the motors so that their thrust passes through the centre of gravity
	has allowed problem-free operation in the 4 DoF framework
\item Use of an external power source allowed the vehicle to be compact and
	lightweight, with the complete system fitting into an above-average sized
	rucksack and being easily transportable by hand.
\end{itemize}

%=====================
\section{Recommendations for future designs}

Despite the general success of the project, several aspects of the design could,
and a few certainly should, be improved if a next iteration or a brand new vehicle
were to be designed. These summarise the lessons learned through the development
and testing of the ROV:
%
\begin{itemize}
\item Making the aft plug removable without the use of bolts or other permanent
	connectors made it very easy to assemble the vehicle on-shore but led to
	the vehicle not being completely watertight. This is a major issue which
	could be addressed by i) using a threaded aft plug with a better seal
	ii) use of a slot-in plug secured with nuts and fitted with additional o-rings
	iii) manufacturing the pressure hull from a less compressible material (steel
	or aluminium) and desigining bespoke watertight connections
\item Removing pre-twist from the tether and making it float would make manoeuvring
	the vehicle far easier
\item Fixing the video latency issue would be necessary for operation in a real
	environment. This may require using a dedicated VGA cable, a different camera,
	or better implementation of the current software
\item At present the motors are placed in water without waterproofing which would not
	be acceptable in the long term in real environments such as seas or rivers.
	A watertight enclosure would be necessary, most likely utilising a magnetic
	coupling, to prevent corrosion of the motors.
\item Being able to redirect a time trace of control inputs and sensor readings to a file
	would make analysing the performance of the vehicle easier
\item In the present design only the internal structure was grounded, extending this
	to include all steel elements would add an important safety feature
\item Using non-linear mapping or having two sensitivity settings for the controller
	should make it easier to perform precise manoeuvres
\item Adding a sensor measuring power output of the battery would be an important
	indicator of the efficiency of the vehicle and would allow planning missions more accurately
\item At present, each of the power wires in the connector in the aft plug (section~\ref{ssection:connectorOnTheROV})
	ends with a~banana connector. Grouping the cables by the corresponding motors
	and using a~single connector for each motor
	would streamline the ROV assembly process significantly.
\item At present, the external steel frame is difficult to disassemble or even
	detach from the central hull due to the way the bolts are situated. At the
	time it was not considered crucial to allow this but having such a possibility
	would allow modifications to the design to be introduced more easily.
\item The electrical system has been significantly over-designed in most places
	in terms of current ratings. However, there are several segments with much lower
	safety factors, which effectively dictate the maximum safe power output from the battery.
	A~careful choice of suitable current ratings for each segment of the design would allow a~much
	more efficient overall system design.
\item Currently used off-the-shelf electronic speed controllers are straightforward
	to utilise but require a~large amount of wiring to operate, which takes up a~considerable
	amount of internal volume of the ROV. Using PCB-mounted variants or even controlling
	the motors directly from the Arduino by pulsing current using transistors could
	allow for a much more compact design. Ultimately, a~small PCB for a~single motor
	or a~pair of motors could be devised. The PCBs could be connected using stack-through
	connectors, thus eliminating most of the wiring used in the current design. This
	would make the system more modular, and lower-mass and thus also smaller and more agile.
\item Having the tether attach to the inspection hatch (aft plug) poses certain difficulties
	during disassembly as often multiple connectors have to be disconnected and reconnected
	just to allow a simple task to be accomplished. Moving the cable port to a stationary part
	of the hull could simplify these operations.
\item Adding controller calibration constants to the GUI settings window would make it more
	easily adjustable by the user.
\end{itemize}
