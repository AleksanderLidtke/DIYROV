\pagestyle{chapter-fancy-style}
\chapter{Introduction}

%=====================
\section{Aim}

The aim of this project was to develop a low-cost remotely operated underwater vehicle (ROV)
for operation in calm, shallow waters.
Its purpose was to allow principle physical and design challenges of building a robotic
underwater platform to be explored first-hand and to highlight possible future paths
for the development of a more refined design.
The intention of the project was not to deliver a state-of-the art vehicle and instead
focus was put on building experience and appreciation for the engineering challenges involved.
The ROV was to be manufactured, as much as possible, from easily available and affordable
off-the-shelf electronics and components. This would allow the devised solutions to
be easily reproducible by anyone with access to the most rudimentary tools and manufacturing
processes, making it easier to share the developed experiences.
The entire project, including software, CAD models, and design calculations is provided
on GitHub at \textcolor{blue}{\url{https://github.com/UnnamedMoose/DIYROV}}.

%=====================
\section{Objectives}

To achieve the specified aim, several objectives have been specified at the outset
of the project:

\begin{enumerate}
\item Create a simple software framework allowing basic electronic components to be controlled from
	a laptop via USB
\item Select a suitable means of propulsion for the ROV
\item Come up with a basic set of sensors which would make it easier to control the vehicle
	and provide basic feedback to the operator
\item Integrate the selected components into a coherent system and devise an arrangement that
	fits into a volume as small as possible
\item Design a watertight vessel to house the electronics using off-the-shelf, cheap DIY products
\item Assemble the ROV and test it in a calm environment such as a swimming pool or a~towing tank
\end{enumerate}

%=====================
\section{Subsequent parts of this document}

The following parts of this informal report will address each of the listed objectives,
and discuss them it in the context of the ultimate aim.
The intention is to convey both the engineering rationale and experience gained throughout the project,
rather than keeping the writing strictly technical. The report is accompanied by
the complete release of the necessary source code (both on-board microcontroller and shore-based computer),
CAD models of the ROV, and design calculations, all of which combined should allow
 the Reader to reproduce the process undertaken at their own leisure.
