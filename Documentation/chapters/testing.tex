\pagestyle{chapter-fancy-style}
\chapter{Testing}

%=====================
\section{Procedure}

The first stage of in-water tests was carried out in the Lamont
towing tank of the University of Southampton. Figure \ref{fig:testingSetup}
shows the experimental set-up used in the described tests.
The towing tank provided a~calm, controlled
environment without currents or wind. The water depth was under \unit[2]{metres},
which was deemed sufficient for the initial test of the pressure vessel.
Furthermore, the window in the side of the towing tank allowed easier observation
of the robot from the side.

There were three main tests to be carried out during the first series of testing:
\begin{enumerate}
\item  Testing of the power output of all the motors and making sure they can drive
	the vehicle in a steady and controlled manner. If not, adjusting the control
	inputs to compensate for any asymmetries
\item Verifying that the provided visual cues, proposed control interface, and
	available motor arrangements are suitable to allow a new user to operate the
	vehicle efficiently and perform simple tasks, such as inspection of underwater
	objects
\item Submerging the vehicle to progressively larger depths to test the water-
	tightness of the outer hull
\end{enumerate}

\begin{figure}[htb]
\begin{center}
\begin{tabular}{c c}
	\subfloat[Dry and manoeuvring tests]
		{\includegraphics*[width=0.45\textwidth,angle=270]{20160826_125304.jpg}
		\label{figure:testingSetup:label:a} } &
	\subfloat[For observation through side window]
		{\includegraphics*[width=0.45\textwidth,angle=270]{20160826_141659.jpg}
		\label{figure:testingSetup:label:b} } \\
\end{tabular}
\end{center}
\caption{View of the testing set-up in the Lamont towing tank.}
\label{fig:testingSetup}
\end{figure}

In order to comply with the health \& safety regulations, a strictly defined
procedure of using the ROV was put in placed and adhered to throughout the testing:
%
\begin{enumerate}
\item Plugging the ROV into a laptop via an Ethernet cable using a USB converter
\item Powering-up the motors using a 12 V DC power supply placed on-shore
\item Testing the camera system and ensuring each of the motors is operational
\item Placing the vehicle in the water using an insulated rod and an attachment point
	provided by the main tether
\item Performing the required tests and tasks
\item Surfacing the vehicle close to the operator and unplugging the power supply
\item Recovering the vehicle by hand and finishing the test
\end{enumerate}

%=====================
\section{Manoeuvring tests}\label{section:manoeuvringTests}

Before the towing tank tests it had been recognised that despite all electric
motors and electronic speed controllers being identical, the same demand signals
from the Arduino board made each of the motors rotate at a slightly different
speed. This had been identified by setting them to the highest RPM in pairs
and listening to the sound they produce while adjusting the tuning constants in order to eliminate the beat frequency (much like tuning a guitar). One of the
vertical thrusters proved to behave particularly differently than the others.
Its tuning parameters were thus adjusted in the Arduino script before the tests,
in order to achieve approximately identical RPM as the other vertical motor
with the same thrust demand.

Once the vehicle was placed in water its balance, initially tested in a small
water container, was verified. The predicted centre of gravity location agreed
well with the expectations, yielding neutral trim. At the outset of the tests
the vehicle was also neutrally buoyant and kept a steady altitude just below the
water surface. Placing the tether attachment at the centre of gravity negated
any moments and prevented the ROV from adopting excess trim angles.
At first, however, the tether was not as buoyant as initially hoped and was dragging
the robot down to the bottom. An impromptu solution was employed whereby sheets
of bubble wrap were attached at even intervals to the tether, making it neutrally
buoyant.

Then, several basic manoeuvres were attempted:
moving in a straight line horizontally and in the vertical plane, orienting the
ROV in a particular direction, and turning while under way.
It has been found that the used motors and propellers provide more than enough
thrust to make the vehicle agile and faster than expected. The control interface
also fulfilled its purpose well and allowed the operator to convey commands to
the ROV with ease. An example of a successful descent manoeuvre is seen in
Figure \ref{fig:rovThroughTheWindow}.
Due to the linear mapping of the joystick displacement to
thrust demand it was rather difficult for the operator to execute very precise
control over the vehicle, however. Such precise control requires very low levels
of thrust, which can only be obtained with minimal joystick displacements in the
current implementation.

It was also found that the tether, in particular the power cable, was so
stiff and twisted (it had spent most of its lifetime on a~reel prior to deployment)
that it made it difficult for the ROV to move freely. This is because the
twist in the tether exerted a~constant moment on the robot, which in turn
rotated the ROV around the tether. This meant that adopting
certain headings was impossible without constant actuation of the thrusters, for example.

It had also been expected that the load on the motors would increase significantly
when they operated in water instead of air, which was indeed observed. At first,
the battery system had been fitted with \unit[1]{A} fuse, which was sufficient
to operate four motors at their maximum thrusts in air. In water, however, 
the \unit[1]{A} fuse allowed the user to only operate the motors at very low RPMs
and made it difficult not to exceed the current limit, thus requiring a~fuse
replacement. In practice, only a~small deviation in the joystick position would
lead to burning the fuse. The internal electronics were capable of
sustaining higher currents and the \unit[1]{A} limit had been chosen somewhat
arbitrarily for increased safety. The safety concern was relaxed during the test
and a~\unit[4]{A} fitted instead. This allowed the user to operate the ROV 
without any restraint while still having a~safeguard built into the system.

\begin{figure}[htb]
\begin{center}
\includegraphics*[width=0.6\textwidth,angle=270]{20160826_135245.jpg}
\end{center}
\caption{ROV floating near the bottom of the towing tank.}
\label{fig:rovThroughTheWindow}
\end{figure}

%=====================
\section{Blind control tests}

Outside of control environment, the operator would not be able to see the ROV from their
workstation and instead they would have to rely solely on the visual and other
cues fed back to them from the on-board sensors. The next stage of tests focused
on simulating this scenario by placing the operator so that the ROV was outside
their field of view. They were then tasked with performing the same tasks as before
(simple manoeuvres - straight line, turning circle, ascent/descent, keeping
a constant heading) when relying only on the video feed from the USB camera
fitted to the ROV.

It has been found that latency present in the visual system was detrimental to
the user's ability to manoeuvre the vehicle easily. They were able to establish
the general direction the ROV was facing in relation to the main features of the
nearby environment. Doing so required them to put in considerable effort and
was further made difficult by the unpredictable forces and moments exerted by
the tether. The exact reason for the existence of latency in the video feed is
unclear - it could be due to the camera itself, due to the same CAT5 cable being
used to transfer video and serial commands to the Arduino, due to some issues
with the OpenCV library and its implementation in the current code, or due to
limitations of the current graphical user interface design.

Furthermore, it has been found that the 640x480 resolution of the camera is not
sufficient to provide detailed view of the ROV surroundings when the vehicle is
put in water with particulates in it. This made it difficult to judge the distance
of the vehicle from the obstacles ahead by using visual cues only.
Nonetheless, enough detail could be captured to allow key environmental
features to be distinguished, as seen in Fig.~\ref{fig:onBoardPictures}.

Lastly, the camera has a~considerable zoom. This makes approaching a~target at 
a~close distance dfficult because it seems to be much larger and thus closer than it in fact is.
At present, none of the ROV components are visible in the camera field of view.
Changing this, for example by using a~camera with a~wider field of view and moving
it further away from the pressure vessel surface, could help the operator judge
proximity to obstacles. This is because both the visible ROV components and the
obstacles would be distorted by the camera in the same way, and the relative
distance could be assessed.

\begin{figure}[htb]
\begin{center}
\begin{tabular}{c c}
	\subfloat[ROV on the free surface]
		{\includegraphics*[width=0.45\textwidth,angle=0]{20160826_135540162340_026.jpg}} &
	\subfloat[Wavemakers at the end of the tank]
		{\includegraphics*[width=0.45\textwidth,angle=0]{20160826_135540162340_539.jpg}} \\
	\subfloat[Operators seen through the side window of the tank]
		{\includegraphics*[width=0.45\textwidth,angle=0]{20160826_140221589085_280.jpg}} &
	\subfloat[The far end of the tank]
		{\includegraphics*[width=0.45\textwidth,angle=0]{20160826_150305045904_012.jpg}} \\
\end{tabular}
\end{center}
\caption{Video frames captured by the on-board camera during towing tank tests.}
\label{fig:onBoardPictures}
\end{figure}

%=====================
\section{Water-tightness tests}

Before the tests in the towing tank, the ROV had been submerged in a small water
container (i.e. a bath tub) for the period of several hours to verify the efficacy
of seals and connections. No water ingress had been observed even for overnight
immersion tests at the depth of approximately 30 cm.

The towing tank tests lasted approximately 3.5-4 hours, during which the ROV
was removed from the water several times. After the initial several seconds
after the first immersion no visible water bubbles were seen escaping from the
hull, indicating the hull remained watertight.
Throughout the tests, however, the vehicle was observed to gradually lose its
neutral buoyancy. Towards the end of the trials, the ROV would linger close to the
tank bottom, and making it ascend using the thrusters alone became difficult.
After the final test run, the aft plug was removed and a~small amount of water was
seen to leak from inside the pressure vessel. The exact volume was not verified due to lack of
experience of the operators. It was sufficient to adversely affect the neutral buoyancy
of the vehicle but did not reach the electronics, which suggests the leak was
small. It is believed the water ingress took place through the aft seal between
the hull and removable aft plug as it was at this location that small air bubbles
were seen escaping at the beginning of each series of tests.
The aft cable port could also be the culprit, although this is seen as less likely.



