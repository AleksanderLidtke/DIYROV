\documentclass[11pt,a4paper,oneside]{report}
\usepackage{units}
\usepackage{booktabs}
\usepackage[hyphens]{url}
\usepackage{tabularx}
\usepackage{color}

%%% PAGE LAYOUT - GENERAL
\setlength{\topmargin}{-0.5cm}
\setlength{\oddsidemargin}{0cm}
\setlength{\evensidemargin}{0cm}
\setlength{\textheight}{24cm}
\setlength{\textwidth}{16cm}

\begin{document}
The steps that need to be taken to complete the first iteration of the ROV:

\begin{enumerate}
\item Figure out how to transmit USB signal over a~long distance (\unit[50]{m}). What we know:
	\begin{itemize}
	\item We need to use \emph{cat5} Ethernet cables.
	\item Need to use USB extenders (can be bought off the shelf \\
		\url{http://www.amazon.co.uk/NEWLink-CAT5-USB-EXTENDER-SYSTEM/dp/B0018D5DUU})
	\end{itemize}
	\textcolor{green}{\textbf{DONE!}}
	
\item Have to minimise the number of wires in the umbilical to make the tether light
	so that it does not restrict the ROV too much and does not sink and get wrapped around stuff on the bottom.
	\begin{itemize}
	\item Should try to use a~USB hub on-board to use only one Ethernet cable to have
		a~live feed from the camera and communicate with the Arduino over serial.
	\item If simultaneous communication with the camera and Arduino can't be achieved over
		the same USB cable, we should use $I^2C$ to communicate with the Arduino - it's simple
		and we will use the same bus when upgrading to AUV to interface the Arduino with
		a~Beagleboard or the Raspberry~Pi that will be responsible for autonomy.
		$I^2C$ range can be extended using P82B96 from Texas~Instruments. \\
			\url{http://uk.rs-online.com/web/p/bus-buffers/8122662/} \\
			\url{http://www.ti.com/lit/ds/symlink/p82b96.pdf} \\
			\url{http://www.nxp.com/documents/application_note/AN10658.pdf}
	\end{itemize}
	\textcolor{green}{\textbf{DONE!}}
	
\item Should use external \unit[9]{} or \unit[12]{V} power supply - portable,
	easy to charge, available: \\
	\url{http://www.ebay.co.uk/bhp/12v-li-ion-battery}
	\\ \textcolor{green}{\textbf{DONE!}}

\item Power management
	\begin{itemize}
	\item Will need MOSFET-type transistors to control power levels delivered to propulsion
		and LED subsystems
	\item Could look at having a power-consumption monitor for future reference and
		design evaluation
	\item Add a voltage monitor for the battery system
	\item Have an emergency breaker in case of a short-circuit maybe?
	\item Need to come up with a way to shift down voltage from 12 to 9
		\url{http://www.ti.com/lit/ds/symlink/lp2950-33.pdf}
	\item need to make it possible to invert poles for the motors to work in reverse
		\url{https://learn.adafruit.com/adafruit-arduino-lesson-15-dc-motor-reversing/parts}
	\end{itemize}

\item Need technical drawings of the Arduino and other components to work on the
	general arrangement (GA)
	
\item Use PVC pipe as the main hull pressure vessel
	\begin{itemize}
	\item Simple pipe like this should do: \\
		\url{http://www.diy.com/departments/floplast-black-soil-pipe-dia110mm-l1m/261999_BQ.prd}
	\item Could use a T-shape as well to have easier access to the central part of the hull
	\item Ends can be capped with socket plugs like this \\
		\url{http://www.diy.com/departments/floplast-terracotta-socket-plug-dia110mm-l123mm/81463_BQ.prd}
	\item External mounting can be done without drilling holes with clamps like so \\
		\url{http://www.diy.com/departments/floplast-black-pipe-clip-dia110mm-l25mm/80964_BQ.prd}
	\item Joining can be achieved using adhesives \\
		\url{http://www.diy.com/departments/evo-stik-pvc-weld-50-ml/36242_BQ.prd}
	\end{itemize}

\item In order to accommodate camera in front a window from plexiglass can be added (may use
	the same adhesive as for the PVC/ABS hull elements) \\
	\url{http://www.plasticsheets.com/5mm-clear-acrylic-sheet/}

\item Cables can be put through threaded plugs
	\begin{itemize}
	\item Can use a male and female elements like these \\
		\url{http://uk.rs-online.com/web/p/pvc-abs-threaded-fittings/2123717/} \\
		\url{http://uk.rs-online.com/web/p/pvc-abs-threaded-fittings/2123537/}
	\item Drill a hole through the male plug, put the cable through and make the
		connector end "stick out" on the inner side of the hull
	\item Use silicone layer to seal-off the hole, fill the rest of the plug with
		epoxy resin for pressure/water resistance (\textbf{make sure the resin does not
		melt the cable})
	\end{itemize}

\item Can use a processor supervisor circuit to provide automatic reset functionality
	from on-board in case the $\mu$-c falls over \\
	\url{http://uk.rs-online.com/web/p/processor-supervisors/6674266/} \\
	\url{http://ww1.microchip.com/downloads/en/DeviceDoc/21370c.pdf}

\item Use ultra-light LEDs to illuminate the camera target \\
	\url{http://www.ebay.com/itm/1000-pcs-PLCC-6-5050-SMD-3-CHIPS-white-Ultra-bright-LED-/231560265873}

\item Tether
	\begin{itemize}
	\item Should consist of an Ethernet CAT5 cable (if using I\textsuperscript{2}C add another one)
	\item Should have a dedicated power cable
	\item Consider adding extra buoyancy elements (Styrofoam) for neutral/slightly positive buoyancy
	\item Should aim to have the tether attached on the top of the ROV, pointing vertically upwards
	\item Consider an extra nylon rope for emergency recovery - make sure to have some slack in the cable
		for stress relief
	\end{itemize}

\item Sensors
	\begin{itemize}
	\item Pressure sensor \\
		\url{http://www.ebay.com/itm/Pressure-transducer-or-sender-60-psi-stainless-steel-for-oil-fuel-air-water-/261508176203}
	\item Temperature sensor (tbc)
	\item Camera - use whatever USB webcam
	\item Gyroscope
	\item Compass
	\item Accelerometer - look at AUV/quadcopter off-the-shelf elements
	\item Think of a way to measure flow speed easily (look at model boat elements,
		Pitot tube, pressure sensors?)
	\end{itemize}
\end{enumerate}

\begin{table}[h]
\caption{Work break-down structure.}
\begin{tabularx}{\textwidth}{@{}lllX@{}}
\toprule
\textbf{Task} 	& \textbf{Person} 	& \textbf{Status} & \textbf{Notes} \\ \midrule
%
$I^2C$        				& Alek  & Done			& Figure out how to send/receive data reliably, re-use serial comms 
													communication functions and data-packeting if possible \\
Component tech. drawings 	& Artur	& Not started	& Create technical drawings of the various
													components being considered to allow early GA of the vessel to be created  \\
USB range extension 		& Both  & Done			& Test the video feed relay over a long USB cable once
													the extenders arrive \\
Masses \& Centres			& Artur	& Not started	& Start indexing the components and noting their
													mass and CG locations for balancing the vessel  \\
Control interface (GUI)		& Both	& In-progress	& Get a way of visualising camera feed and sending/receiving info \\
%
\bottomrule
\end{tabularx}
\end{table}

\end{document}
